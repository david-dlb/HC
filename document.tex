\documentclass[]{article}
\usepackage[spanish]{babel}
\usepackage{indentfirst}
%opening
\title{La evolución de los microprocesadores y su impacto tecnológico}
\author{David Lezcano Becerra \\ Facultad de Matematica y Computacion \\ Universidad de la Habana }
\author{Albaro Suárez Valdes \\ Facultad de Matematica y Computacion \\ Universidad de la Habana }


\begin{document}

\maketitle

\begin{abstract}
Este trabajo explora la evolución histórica de los microprocesadores, desde el Intel 4004 hasta las modernas arquitecturas multicore. Se analiza cómo cada avance tecnológico - desde los tubos de vacío hasta los circuitos integrados - permitió nuevas aplicaciones computacionales. El documento destaca hitos como el primer microprocesador de 4 bits, la transición a arquitecturas de 32 y 64 bits, y el desarrollo de procesadores especializados para inteligencia artificial. Se examina el impacto social de esta evolución, incluyendo la democratización del acceso a la computación y los cambios en la ciencia, educación y comunicación. Finalmente, se discuten las perspectivas futuras con tecnologías emergentes como la computación cuántica y los procesadores neuromórficos.
\end{abstract}

\textbf{Palabras clave:} microprocesador, Intel 4004, evolución tecnológica, computación cuántica, arquitectura x86, Ley de Moore

\section{Introducción histórica}
Los microprocesadores surgieron de la evolución de tecnologías previas como los tubos de vacío y los transistores discretos\cite{Steve_Jobs}. El primer microprocesador comercial, el Intel 4004\cite{Steve_Jobs}, marcó en 1971 el inicio de una revolución que transformaría la sociedad.

\subsection{Predecesores tecnológicos}
Antes del microprocesador, las computadoras utilizaban:
\begin{itemize}
\item Tubos de vacío (1940s-1950s)
\item Transistores discretos (1950s-1960s)
\item Circuitos integrados SSI/MSI (1960s)
\end{itemize}

\section{Evolución tecnológica}

\subsection{Primeras generaciones (1970s)}
El Intel 4004 (1971) contenía 2,300 transistores y operaba a 740 kHz. Su desarrollo para calculadoras demostró el potencial de integrar una CPU completa en un chip\cite{Steve_Jobs}.

\subsection{Arquitecturas de 16 y 32 bits (1980s)}
El Intel 80286 (1982) introdujo:
\begin{itemize}
\item Memoria virtual
\item Protección de memoria
\item Hasta 16 MB de espacio de direcciones
\end{itemize}

\section{Impacto social y científico}

\subsection{Democratización de la computación}
La reducción de costos permitió:
\begin{itemize}
\item Computadoras personales (1980s)
\item Dispositivos móviles (2000s)
\item Internet de las cosas (2010s)
\end{itemize}

\subsection{Aplicaciones científicas}
Los microprocesadores modernos permiten:
\begin{itemize}
\item Simulaciones climáticas
\item Investigación genómica
\item Modelado de proteínas
\end{itemize}

\section{Tecnologías emergentes}

\subsection{Computación cuántica}
Procesadores como IBM Quantum muestran potencial para resolver problemas intratables para computación clásica\cite{Steve_Jobs}.

\subsection{Procesadores neuromórficos}
Arquitecturas como Intel Loihi imitan el funcionamiento del cerebro humano para mayor eficiencia energética.

\section{Conclusiones}
La evolución de los microprocesadores ha seguido la Ley de Moore durante cinco décadas, permitiendo avances impensados en computación personal, científica e industrial. Los desafíos futuros incluyen superar límites físicos y desarrollar nuevas arquitecturas para la era post-Moore.

\bibliographystyle{plain}
\bibliography{bibliografia}

\end{document}